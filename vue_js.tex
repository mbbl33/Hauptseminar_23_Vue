%! Author = biebl
%! Date = 28.04.2023

\chapter{Vue.js}
In diesem Kapitel möchte ich genauer auf Vue.js eingehen.
Dabei möchte ich ausführlich auf Vue.js und seine Ziele eingehen.
Weiterhin möchte ich auf die Historie und Meilensteine in der Entwicklung von Vue.js erläutern.


\section{Was ist Vue.js?}
Bei Vue.js oder einfach nur Vue genannt handelt es sich um ein progressives JavaScript Frontend Framework.
Progressive heißt in diesem Zusammenhang,
dass es sich an die Bedürfnisse der Entwickler anpassen lässt.
Der Kern von Vue kümmert sich ausschließlich um den View-Layer und ist rund 20 KB groß.
Für weitere benötigte Features kann Vue.js modular erweitert werden. \cite[S. 523-524]{bin2019}
\\
Das Konzept von Vue ermöglicht es Vue in bestehende Projekte integriert zu werden
und soll es leichter machen ein Projekt
schrittweise von anderen Frameworks zu Vue.js migrieren zulassen  \cite[S. 1]{peterke2019}.

\section{Geschichte}

\begin{figure}[!htb]
    \centering
    \includegraphics[width=.3\textwidth]{img/you}
    \caption{Evan You}
    \label{fig:you}
\end{figure}

Erschaffer und Projektleiter von Vue.js ist der gebürtige Chinese Evan You (Abb. \ref{fig:you}).
Evan You arbeitete nach seinem Studium zunächst bei Google und später bei Meteor wo er
an Meteor.js beteiligt war.
Das Vue.js-Projekt startete er im Juli 2013 mit der Absicht \emph{ ein kleines Angular}
zuschreiben.
Während seiner Zeit bei Google hatte er viele Projekte mit Angular und
sein Ziel mit Vue war es zunächst, die seiner Meinung nach schlechten Sachen von Angular herauszufiltern
und die seiner ansicht nach guten Dinge auf Vue zu übertragen.



%linked in
%YouTube Vortrag https://www.youtube.com/watch?v=p2P3z7p_zTI&