%! Author = biebl
%! Date = 02.05.2023

\chapter{Technische Details}
In diesem Kapitel soll es um die technischen Details von Vue.js gehen.
Es wird ein Überblick gezeigt von verschiedenen in Vue.js vorhandenen Elementen
und es wird auf diese anhand von Beispielen genauer eingegangen.

\section{Options API vs. Composition API}\label{sec:options-api-and-composition-api}
Vue organisiert einzelne Komponenten als \emph{Single-File Components (SFC)},
die im HTML-ähnlichen Dateiformat mit der Dateiendung \lstinline{.vue} vorliegen.
In einem SFC werden Aussehen, Struktur und Logik einer Komponente gebündelt, bestehend aus HTML-, CSS- und JavaScript-Elementen.
Für den Aufbau einer solchen SFC gibt es seit Vue 3 zwei verschiedene Möglichkeiten.
Zum einen gibt es die \emph{Options API} und zum anderen die \emph{Composition API}.
\subsection*{Options API}
Die länger vorhandene Variante ist die Options API.
Bei der Options API wird für die Komponente ein JavaScript Objekt angelegt.
Das Objekt kann verschiedene Optionen enthalten, darunter \lstinline{data} für Daten, \lstinline{methods} für Methoden und auch Lifecycle hooks wie \lstinline{mounted}.

\subsection*{Composition API}
Die Composition API kam mit Vue 3 hinzu und wurde später für Vue 2 nachgereicht \cite{vueFAQ}.
Bei der Verwendung der Composition API wird in einem \lstinline{import}-Statement
die benötigten API-Features wie zum Beispiel Lifecycle hooks angegeben.
Der Code bei verwendung der Composition API wird in der Regel zwischen einem \lstinline{<script setup>} Tag verwendet.
Mit dem Tag wird angegeben, dass zur Compilezeite eine Transformation durch geführt werden soll,
die Composition API mit weniger Redundanz im Code zu nuten.
So können Variablen und Funktionen direkt im Template genutzt werden.

\subsection*{Gegenüberstellung}
%TODO https://vuejs.org/guide/introduction.html#api-styles

% ----------------------------------------------------------------------------