%! Author = biebl
%! Date = 12.05.2023


\chapter{Fazit}
Die in Abschnitt \ref{sec:geschichte} von Evan You definierten Ziele für Vue.js sind meines erachtens erfüllt.
Die Absicht ein schlankes Framework zu erstellen ist erfüllt und wird deutlich beim Vergleichen der Speicherauslastung
der zwei äquivalenten Einkaufslistenapps.
Die Dynamik von Vue.js wird durch die Modulare erweiterbarkeit und meines erachtens durch das
Konzept der SFC deutlich.
Während ich in Angular immer wieder zwischen drei Dateien springen muss, um eine Komponente zu bearbeiten,
ist in Vue.js alles zentral in einer Datei.
Die ähnliche Umsetzung einiger Konzepte wie das Routing oder die Verwendung von Direktiven machen
den wechsel zwischen Angular und Vue.js einfacher.
Für kleinere Anwendungen ist für mich Vue.js aufgrund der bereits genannten Konzepte vorzuziehen.
Wie sich Vue.js im Vergleich zu Angular bei größeren Projekten schlägt, lässt sich aus dem Vergleich
in Abschnitt \ref{ch:vergleich-zu-angular-anhand-einer-einfachen-app} nicht eindeutig sagen.
Der Vergleich in Abschnitt \ref{ch:vergleich-zu-angular-anhand-einer-einfachen-app} ist für Rückschlüsse
auf umfangreichere Projekte ungeeignet, da eine Betrachtung von Aspekten wie zusammenspiel mit einem Backend
oder Betrachtung des Featureumfangs fehlen.
Der Vergleich in Abschnitt \ref{ch:vergleich-zu-angular-anhand-einer-einfachen-app} bringt viel mehr einen
Einblick von Vue.js für Leute die bereits Erfahrung mit Angular haben.
Abschließend kann ich sagen Vue.js ist das, was es sein möchte, eine schlankere Alternative zu Angular.
