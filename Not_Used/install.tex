% ----------------------------------------------------------------------------
% Copyright (c) 2016 - 2020 by Burkhardt Renz. All rights reserved.
% Die Vorlage für eine Abschlussarbeit in der Informatik am Fachbereich
% MNI der THM ist lizenziert unter einer Creative Commons
% Namensnennung-Nicht kommerziell 4.0 International Lizenz.
%
% Id:$
% ----------------------------------------------------------------------------

\chapter{Installation von \LaTeX}

Der Text dieses Anhangs steht in \verb=install.tex=.

\section{Windows}

\subsection*{Installationsschritte}

\begin{enumerate}
	\item Downloadseite von MiKTeX \url{http://www.miktex.org/download}.

	\item Die Basisversion \textbf{Basic MiKTeX 2.9 Installer} herunterladen.

	\item Die Installationsdatei namens \verb=basic-miktex-2.9.xxxx.exe=  
		starten.
		
	\item Dem Installationswizard folgen (am einfachsten die
		vorgeschlagenen Werte für Verzeichnisse usw. übernehmen).

	\item	Einen Ordner für die eigenen Dokumente erstellen, z.B. in \emph{Eigene
Dokumente}.

\end{enumerate}

\subsection*{Erste Schritte mit \TeX works}

\begin{enumerate}
	\item Im Startmenü oder den Apps das Programm \emph{TeXworks} suchen
		und starten. Optional: Zur späteren Bequemlichkeit das Programm an die
		Taskleiste anheften. 

		Es erscheint das Editierfenster auf der linken Hälfte des Bildschirms.

	\item Zum ersten Ausprobieren im Menü \emph{File} den Unterpunkt \emph{New 
	 	from Template} auswählen und in dem dann erscheinenden Dialog \emph{Basic 
		LaTeX documents} und \emph{article.tex}

	\item	Die Datei wird im Editorfenster geöffnet. Um daraus das PDF-Dokument
		zu erstellen drückt man auf den grünen Button links oben.

		Nun öffnet sich ein Dialog zum Abspeichern des Dokuments. Danach:

	\item	Die LaTeX-Datei wird übersetzt. Im linken Fenster unten sieht man die
		Meldungen über den Fortschritt dieses Vorgangs. (Beim ersten Mal
		dauert das relativ lange, weil diverse Pakete für LaTeX aus dem
		Internet heruntergeladen werden.)

	\item	Nach einiger Zeit erscheint das Ergebnis in einem neuen Fenster auf
		der rechten Seite des Bildschirms.

	\item	Jetzt kann es losgehen: links editieren, Erstellen des Dokuments
		mit dem grünen Button starten und rechts das Ergebnis überprüfen.
\end{enumerate}


\section{Mac OSX}

\subsection*{Installation von Mac\TeX}

\begin{enumerate}
	\item Das Package MacTeX.pkg erhältlich bei
		\url{http://www.tug.org/mactex/} herunterladen.
	\item Öffnet man das Paket mit Doppelklick, startet die Installation
		und sie wird schrittweise durchgeführt.	
\end{enumerate}

\subsection*{Arbeiten mit TeXShop}

Nach der Installation hat man im Launchpad eine App namens TeXShop. Dies
ist ein Editor für \TeX\ und \LaTeX.

\section{Linux}

Die einfachste Variante besteht darin, eine komplette \TeX
Live-Installation durchzuführen mittels

\begin{lstlisting}
sudo apt-get install texlive-full
\end{lstlisting}

Dabei wird allerdings vieles installiert, das man voraussichtlich
niemals braucht. Andererseits muss man aber nicht wissen, was man
mindestens installieren muss, damit \LaTeX\ verwendet werden kann.

Auf Github findet man Skripte, die Installationen z.B. unter Ubuntu
steuern können. Beispiel:
\url{https://github.com/scottkosty/install-tl-ubuntu}. Ich habe aber
keine Ahnung, wie gut diese Skripte tatsächlich sind.

% ----------------------------------------------------------------------------
