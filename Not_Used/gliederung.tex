% ----------------------------------------------------------------------------
% Copyright (c) 2016 -2020 by Burkhardt Renz. All rights reserved.
% Die Vorlage für eine Abschlussarbeit in der Informatik am Fachbereich
% MNI der THM ist lizenziert unter einer Creative Commons
% Namensnennung-Nicht kommerziell 4.0 International Lizenz.
%
% Id:$
% ----------------------------------------------------------------------------

\chapter{Elemente für die Gliederung}

Der Text dieses Kapitels steht in \verb=gliederung.tex=. Er bezieht sich
auf \verb=vorlage.tex= und \verb=gliederung.tex=.

\section{Teile}

Eine Abschlussarbeit besteht aus drei großen Teilen:

\begin{itemize}
	\item Dem Vorderteil (\verb=frontmatter=) mit
		\begin{itemize}
			\item der Titelseite,
			\item der eidesstattlichen Erklärung,
			\item der Zusammenfassung,
			\item dem Inhaltsverzeichnis und
			\item den Verzeichnissen von Abbildungen, Tabellen und Listings,
		\end{itemize}
	\item dem Hauptteil (\verb=mainmatter=) mit den Kapiteln der Arbeit
		und
	\item dem Anhang (\verb=backmatter=) mit Anhängen und dem
		Literaturverzeichnis	
\end{itemize}

Manchmal erwarten Dozentinnen oder Dozenten, dass eine Abschlussarbeit
Verzeichnisse der Abbildungen, Tabellen und Listings oder auch ein Glossar
der verwendeten Begriffe enthält. Deshalb enthält unsere Vorlage diese
Verzeichnisse. Ein Glossar kann man mit der Umgebung \verb=description=
oder \verb=labeling= erzeugen, die in Abschnitt
\ref{sec:stichwortlisten}  beschrieben werden.

Ich persönlich finde, dass man auf diese Verzeichnisse verzichten kann.
Und ein Glossar ist meines Erachtens nur nötig, wenn man viele
Fachbegriffe in der Arbeit verwendet, die einem in der Informatik
Kundigen nicht geläufig sind.

\section{Kapitel und ihre Untergliederungen}

Die Dokumentklasse \verb=scrbook= hat als Möglichkeiten der Gliederung
zunächst den Teil (\verb=part=). Er wird in diesem Dokument nicht
verwendet und meistens ist eine Bachelor- oder Masterarbeit nicht so
umfangreich, als dass man sie in Teile unterteilen müsste.

Die nächste Ebene ist das Kapital (\verb=chapter=), wie wir auf der
vorigen Seite den Anfang eines solchen sehen. Kapitel beginnen immer auf
einer rechten Seite.

Dann kommt der Abschnitt (\verb=section=) --- in einem solchen befinden wir
uns im Moment.

\subsection{Unterabschnitt}

Dies ist eine \verb=subsection=.

\subsubsection{Unterunterabschnitt}

Jetzt sind wir noch eine Ebene tiefer, in der \verb=subsubsection=. Eine
Gliederung sollte ausgewogen sein, auch in Bezug auf die
Gliederungstiefe. Deshalb sollte man eher nicht bis zum
Unterunterabschnitt gehen.


\paragraph{Absatz mit Überschrift}

Dies ist ein Absatz mit Überschrift \verb=paragraph=. Die Überschrift
wird im Dokument hervorgehoben, hat aber keine eigene Zeile. Typografen
nennen das auch einen \enquote{Spieß}.

\subparagraph{Unterabsatz}

Dies ist ein Unterabsatz \verb=subparagraph=, auch ein \enquote{Spieß}.

\section{Untergliederungen im Text}

\subsection{Aufzählungen}

Man kann nummerierte Aufzählungen mit der Umgebung \verb=enumerate=
erzeugen. Dies eignet sich für Aufzählungen, die eine Reihenfolge haben
und ist oft einer Aneinanderreihung im Text vorzuziehen, weil
übersichtlicher.

Eine Bachelorarbeit besteht aus

\begin{enumerate}
	\item einem ersten Kapitel
		\begin{enumerate}
			\item einem ersten Abschnitt darin,
			\item einem zweiten Abschnitt darin,
		\end{enumerate}
	\item einem zweiten Kapitel
	\item usw. usf.	
\end{enumerate}

Aufzählungen, die keine inhaltliche Reihenfolge haben, kann man mit der
Umgebung \verb=itemize= darstellen:

\begin{itemize}
	\item eine wichtige Aussage
	\item noch eine wichtige Aussage
		\begin{itemize}
			\item mit einer Ausprägung
			\item und noch einer Ausprägung	
		\end{itemize}
	\item \dots
\end{itemize}

\subsection{Stichwortlisten}
\label{sec:stichwortlisten}

Ein Beispiel für die Umgebung \verb=description= habe ich aus der
Dokumentation von \textsf{KOMA-Script} übernommen:

\begin{description}
	\item[empty] ist der Seitenstil, bei dem Kopf- und Fußzeile vollständig 
		leer bleiben. 
	\item[plain] ist der Seitenstil, bei dem keinerlei Kolumnentitel verwendet wird. 
	\item[headings] ist der Seitenstil für automatische Kolumnentitel. 
	\item[myheadings] ist der Seitenstil für manuelle Kolumnentitel. 
\end{description}

Es gibt auch noch die Umgebung \verb=labeling=, bei der man durch ein
Muster angeben kann, wie breit die Einrückung ist. Im Beispiel mit den
Seitenstilen könnte man das so machen:

\setkomafont{labelinglabel}{\sffamily\bfseries}
\begin{labeling}{myheadings}
	\item[empty] ist der Seitenstil, bei dem Kopf- und Fußzeile vollständig 
		leer bleiben. 
	\item[plain] ist der Seitenstil, bei dem keinerlei Kolumnentitel verwendet wird. 
	\item[headings] ist der Seitenstil für automatische Kolumnentitel. 
	\item[myheadings] ist der Seitenstil für manuelle Kolumnentitel. 
\end{labeling}

\section{Zum Literaturverzeichnis}

In der Vorlage findet man das Literaturverzeichnis in der Datei
\verb=litverz.tex=. 

Das Literaturverzeichnis kann man in \LaTeX\ auf zwei Arten erstellen:

Die einfache Variante besteht darin, dass man die Umgebung
\verb=bibliography= verwendet. So haben wir das in diesem Dokument
gemacht. Die Angaben zur Literatur stehen in \verb=litverz=. Jeder
Eintrag hat einen \emph{key}, den wir im Text im Befehl \verb=\cite=
referenzieren können. Die Referenz wird dann als Nummer im Text
angegeben und die entsprechende Nummer erscheint auch im
Literaturverzeichnis.

Die etwas aufwändigere Variante besteht darin, \verb=bibtex= zu
verwenden. Dies lohnt sich insbesondere dann, wenn man in verschiedenen
Manuskripte immer wieder dieselben Literaturverweise verwendet.

Mit \verb=bibtex= speichert man die Literaturangaben in einer Art
Datenbank, einer BibTeX-Datei. Die im Manuskript referenzierten Arbeiten
werden dann von \verb=bibtex= automatisch in das Literaturverzeichnis
übernommen. Die Formatierung des Verweises im Text und der Einträge im
Literaturverzeichnis wird dabei durch eine eigene Datei, dem
BibTeX-Stil, gesteuert. Will man die Verweise durch Nummern, wählt man
als Stil \verb=plain=, will man die Verweise durch die Abkürzung des
Autorennamens mit Angabe des Jahres, wählt man z.B. den Stil
\verb=alpha=. Es gibt eine Vielzahl solcher BibTeX-Stile, siehe
\url{https://www.ctan.org/tex-archive/biblio/bibtex/contrib}. Man kann sogar
eigene BibTeX-Stile definieren mit \verb=tex makebst=.

Mehr über BibTeX findet man z.B. bei \url{https://de.wikibooks.org/wiki/LaTeX-Kompendium:_Zitieren_mit_BibTeX}.

% ----------------------------------------------------------------------------
